\section{Сложная функция. Конформность. Голоморфность. ДЛО}
Вводили частные производны каким-то странным образом. Покажем, что $\CP{}z$ и $\CP{ }{\ol z}$ ведут себя так, будто бы $z$ и $\ol z$ независимые переменные.
\begin{Pre}
	Пусть $f$ и $g$ являются $\R$-диффенцируемыми в точке $z_0$. Тогда $f\pm g$, $f\cdot g$, $f/g$ (если $g(z_0)\ne 0$) тоже являются $\R$-дифференцируемыми. Причём
	\[
		\CP{(f\cdot g)}z\bigg|_{z_0} = \CP fz\bigg|_{z_0} g(z_0) + f(z_0)\CP gz\bigg|_{z_0},
	\] аналогично для $\CP{ }{\ol z}$.  А также
	\[
		\CP{(f/g)}z\bigg|_{z_0} = \frac{\left.\CP fz\right|_{z_0}g(z_0) - \left.f(z_0)\CP gz\right|_{z_0}}{\big(g(z_0)\big)^2},
	\]
	аналогично для $\CP{ }{\ol z}$.
\end{Pre}
\begin{Sl}
	Если $f$ и $g$ являются $\C$-дифференцируемыми, то $f\cdot g$ является $\C$-дифференцируемой и верна формула для комплексной производной.
\end{Sl}

Теперь посмотрим на сложную функцию.
\begin{Pre}
	Пусть $g$ является $\R$-дифференцируемой в точке $w_0 = g(z_0)$. Тогда $h(z) = f\big(g(z)\big)$ является $\R$-дифференцируемой в $z_0$, причём
	\[
		\CP hz\bigg|_{z_0} = \CP fw\bigg|_{w_0}\cdot \CP gz\bigg|_{z_0} + \CP f{\ol w}\bigg|_{w_0}\cdot \CP{\ol g}z\bigg|_{z_0}.
	\]
\end{Pre}
\begin{Sl}[Теорема о комплексной производной сложной функции]
	Пусть $f,g$ $\C$-дифференцируемы. Тогда $f\circ f$ $\C$-дифференцируема и $h'(z_0) = f'(w_0)\cdot g'(z_0)$.
\end{Sl}
\begin{Zam}
	Таблица производных, как обычно.
	\begin{itemize}
		\item $\brancH{z^p}\equiv = e^{p\ln z}$. Если $p\in\Z$, всё однозначно. В общем случае диффенцируема основная ветвь:
			\[
				(\brancH{z^p})' = e^{p\ln z}\frac1z\cdot p = \brancH{z^p}\frac pz.
			\]
		\item $(e^z)'=e^z$;
		\item $(\cos z)' = -\sin z$;
		\item $(\sin z)' = \cos z$;
		\item $(\tg z)' = \frac1{\cos^2 z}$;
		\item $(\ctg z)' = \frac{-1}{\sin^2 z}$;
		\item $(\ln z)' = \frac1z$.
	\end{itemize}
\end{Zam}
Это надо всё вывести.
\begin{Zam}
	Достаточным условием дифференцируемости функции двух вещественных переменных является непрерывность частных производных.
\end{Zam}
\begin{Def}
	Пусть $f$ является $\R$-диффенцируемой в точке $z_0$. Функция $f$ называется конформной (сохраняющей форму) в точке $z_0$, если её дифференциал в точке $z_0\pau df|_{z_0}(\Delta z)$ является композицией гоотетии с положительным коэффициентом и поворота, оба с центром $\Delta z = 0$, то есть
	\[
		df|_{z_0}(\Delta z) = k e^{i\,\theta}\Delta z,\pau k = k_{z_0}>0,\ \theta = \theta_{z_0}\in\R.
	\]
	Здесь $k,\theta$ не зависят от $\Delta z$.
\end{Def}
\begin{Ut}
	$f$ является конформной в $z_0$, если и только если $f$ является $\C$-дифференцируемой в точке $z_0$ и $f'(z_0)\ne0$.
\end{Ut}
На самом деле $f'(z_0) = k e^{i\,\theta}$, то есть $k = \big|f'(z_0)\big|>0$, а $\theta \in\Arg f'(z_0)$. В этом состоит геометрический смысл комплексной производной, если она не равна нулю.

Например, $f(z) =e^z$. Тогда $f'(z_0) = e^{z_0}$. Здесь $k = e^{x_0}$, $\theta = y_0\pmod{2\pi}$. Значит, на мнимой оси нет растяжения, а на прямых $y = 2\pi k$ нет поворота.

\begin{Def}
	Функция называется локально конформной в области $D\subset \C$, если $f$ является конформной в каждой точке $z_0\in D$, то есть $\forall\ z_0\in D\pau \exists\ f'(z_0)\ne 0$.
\end{Def}
\begin{Def}
	Функция $f$ называется конмормной в области $D$, если $f$ локально конформна и взаимнооднозначна в $D$.
\end{Def}
Например, $f(z) = z^2$. $f'(z) = 2z$, значит, $f$ локально конформна в $\C_* = \C\dd\{0\}$. $z^2$ склеивает $z$ и $-z$, но в любой полуплоскости, у которой граница содержит ноль, является конформной.

Ещё пример: $e^z$ локально конформна в $\C$. Просто конформна в $\Pi_{(\alpha,\alpha+2\pi)}$. Такие области называются максимальными областями конформности для $e^z$.
\begin{Def}
	Функция $f$ называется голоморфной (простой формы) в точке $z_0$, если $f$ является $\C$-дифференцируемой в некоторой окрестности точки $z_0$.
\end{Def}
\begin{Def}
	Функция $f$ называется голоморфной в области $D\subset \C$, если $f(z)$ всюду в $D$ имеет комплексную производную. Класс всех голоморфных функций в области $D$ будем обозначать $\A(D)$.
\end{Def}
Функции $e^z$, $\sin z$, $\cos z$, многочлены $p(z)$ голоморфны в $\C$. Функции, голоморфные на всей комплексной плоскости называют целыми.

Например, $f(z) = z\ol z = x^2+y^2$ "--- $\R$-дифференцируемая функция, а $\CP f{\ol z} = z$. Значит, $z=0$ "--- единственная точка $\C$-дифференцируемости, причём $f'(0) = 0$. Таким образом, эта функция нигде не конформна и нигде не голоморфна.
\subsection{Дробно-линейные отображения}
\begin{Def}
	ДЛО "--- это функции вида
	\[
		\Lambda(z) = \frac{a\,z + b}{c\,z + d},
	\]
	где $a,b,c,d\in\C$ и $\begin{vmatrix}
		a&b\\c&d
	\end{vmatrix}\ne0$, то есть $\Lambda\ne\const$.
\end{Def}

Обсудим свойства такого отображения.
\begin{Ut}
	Любое ДЛО является гомеоморфизмом $\ol{\C}$ на $\ol{\C}$, где $\ol{\C} = \C\cup\{\infty\}$ "--- расширенная комплексная плоскость, представляющаяся, как сфера Римана.
\end{Ut}
Когда $z\to z_0\in\C$, всё как раньше. Оставшийся случай $z\to\infty\in\ol\C\iff |z|\to+\infty$. Символу $\infty$ соответствует полюс сферы.

Если $c=0$, то $a\ne 0$ и $d\ne0$. Можно считать, что $d=1$. Тогда $\Lambda(z) = \underbrace{a}_{k e^{i\,\theta}}\left(z + \frac ba\right)$ и считаем $\Lambda(\infty) = \infty$.

Пусть далее $c\ne0$. Тогда есть особая точка $z_0 = -\frac dc\in\C$. Полагаем $\Lambda(z_0)=\infty$, $\Lambda(\infty) = \frac ac$.
\begin{Task}
	Проверить, что указанное отображение является гомеоморфизмом.
\end{Task}
\begin{Ut}
	Все ДЛО образуют группу относительн комфозиции.
\end{Ut}
\begin{Ut}
	Любое ДЛО конформно отображает $\ol\C$ на $\ol\C$.
\end{Ut}
\begin{Proof}
	Считаем производную
	\[
		\Lambda'(z) = \frac{a(cz + d) - c(az+b)}{(cz+d)^2} = \frac{\begin{vmatrix}
			a & b\\ c& d
		\end{vmatrix}}{(cz + d)^2}
	\] при $z\ne z_0$.
\end{Proof}
Таким образом, ДЛО сохраняют углы между гладкими кривыми.
\begin{Ut}[круговое свойство]
	Любое ДЛО обобщённую окружность переводит в обобщённую окружность.
\end{Ut}
Обобщённая окружность в $\ol\C$ "--- это обыкновеная окружность или прямая, объединённая с $\infty$.

При $c\ne 0$ считаем $c=1$ можно ДЛО представить в виде $\Lambda(z) = \frac{\gamma}{z-z_0} + b$. В данном случае $\Lambda$ является композицией
\begin{itemize}
	\item сдвига $z_1 = z - z_0$;
	\item симметрии и инверсии $z_2 = \frac1{z_1}$;
	\item гомотетии $z_3 = a z_2$;
	\item и поворота $z = z_3 + b$.
\end{itemize}
Таким образ, доказывать круговое свойство достаточно для отображения $\frac1z$. Остальные этапы очевидны.
