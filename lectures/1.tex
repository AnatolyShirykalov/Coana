Вспомним, что такое комплексные числа $\C$. Это никакая не мистика. Всё вводим аксиоматически.
\begin{Def}
	$\C = \{z = x+ i\,y\mid x,y\in\R,\ i$ "--- символ$\}$\footnote{Запись $z = x+i\,y$ называют алгебраической формой комплексного числа. $x = \Re(z)$ "--- вещественная часть $z$, $y = \Im(z)$ "--- мнимая часть. Символ $i$ играет роль базисного вектора.}, где введены операции
	\begin{itemize}
		\item[«$+$»] $z_1+ z_2 = (x_1+x_2) + i\,(y_1+y_2)$;
		\item[«$\cdot$»] $z_1\cdot z_2 = x_1\,x_2 - y_1\,y_2 + i\,(x_1\,y_2+x_2\,y_1)$,
	\end{itemize}
	где $z_{1,2} = x_{1,2} + i\, y_{1,2}$. $z_1$ и $z_2$ различны, если и только если $x_1\ne x_2$ и $y_1\ne y_2$.
\end{Def}
\begin{Task}
	Доказать, что $\C$ "--- поле. (Нулём называют $0+i\,0$, единицей $1 + i\,0$, обратным $\frac1z \hm= \frac{\ol z}{z\ol z} \hm= \frac{\ol z}{x^2+y^2} \hm=:\frac{\ol z}{|z|^2} \hm= \frac{x-i\,y}{x^2+y^2}$.)
\end{Task}
\begin{Def}
	$\ol z = x - i\,y$ называется сопряжённым к $z$.
\end{Def}
Совокупность вида $\{x+i\,0\mid x\in\R\}$ "--- это подполе в $\C$, изоморфное $\R$. Будем отождествлять $x+i\,0$ и $x$, $0+i\, y$ с $i\,y$.

Посмотрим, чему равняется $i^2 = (0 + i\, 1)^2 = -1$.

Хорошо бы это поле уметь интерпретировать по крайней мере геометрически: это просто декартова система координат на плоскости.
\begin{figure}[H]\centering
	\begin{tikzpicture}[scale=0.5]
		\coordinate (z) at (-2.2,2.4);
		\coordinate (o) at (0,0);
		\draw[->] (-6,0) -- (6,0) node[anchor=north] {$x$};
		\draw	(0,0) node[anchor=north east] {0}
			(2,0) node[anchor=north] {$1$}
			(z |- o) node[anchor=north] {$x$}
			(o |- z) node[anchor=west] {$i\,y$};
		\draw[->] (0,-2) -- (0,4) node[anchor=east] {$i\,y$};
		\draw	(0,2) node[anchor=east]{$i$};
		\draw[dashed](z |- o) -- (z) -- (o |- z)
			      (o) -- (z);
 	        \fill[black] (z) circle (2pt) node[anchor=south east]{$z$};
		\draw[->] (1,0) arc (0:130:1cm);
		\draw (1,1) node{$\phi_0$};
	\end{tikzpicture}
	\caption{Геометрическая интерпретация}
	\label{pic1}
\end{figure}
Каждая $z$ может быть представлена в виде радиус-вектора в декартовой плоскости: $z = x+ i\, y\leftrightarrow \{x,y\}\leftrightarrow(x,y)$, где $\{x,y\}$ "--- радиус-вектор, а $(x,y)$ "--- точка плоскости. Вектора мы далее будем обозначать фигурными скобками.

Пока мы не умножаем комплексные числа, мы имеем дело с обычным линейным пространством. Геометрическая интерпретация подсказывается ещё одну форму записи $z$.
\subsection{Тригонометрическая форма записи}
Пусть $z\ne 0$. Возьмём положительное направление вещественной оси ($ox$) и в ближайшем к $\{x,y\}$ направлении повернём на угол $\phi_0\in(-\pi,\pi]$. Углы у нас считаются в радианах.

С этого момента никогда $i$ не будет индексом, а углы всегда будут в радианах.
\begin{Task}
	$\pi$ иррационально. (Полторы странички за три часа. Кто захочет, расскажу, где прочитать.)
\end{Task}
\begin{Def}
	Главным значением (полярного) аргумента\footnote{Это плохое название. У функции потом само $z$ будет аргументом.} числа $z$ называется $\arg(z) = \phi_0$.

	Совокупным полярным аргументом числа $z$ называется множество $\Arg(z) = \{\phi_0+2\pi k,\ k\in \Z\}$, а $\phi\in \Arg(z)$ называется представителем.
\end{Def}
Пишем, как точку в полярных координатах, $z = r \cos\phi + i\, r\sin \phi = r(\cos\phi + i\, \sin\phi)$, последнее выражение называется тригонометрической формой числа $z$, $r = |z| = \sqrt{x^2+y^2}$ называется модулем числа $z$.

Например, пусть $z = 1+ i$. Тогда а тригонометрической форме $z = \sqrt{2}\left(\cos\frac\pi4 + i\, \sin\frac\pi 4\right)$.
\begin{Ut}
	Пусть $z_{1,2}\ne0$ и $z_{1,2} = r_{1,2}(\cos\phi_{1,2} + i\,\sin\phi_{1,2})$. Тогда 
	\[
		z_1\cdot z_2 = r_1\cdot r_2 \big(\cos(\phi_1+\phi_2) + i\,\sin(\phi_1+\phi_2)\big).
	\]
\end{Ut}
Из этого утверждения вытекает ассоциативность умножения.

Оказывается, что именно такое умножение является важным среди для обобщения умножения в $\R$. Один из замечательных примеров, когда появляются комплексные числа. Рассмотрим функцию $\frac1{1+x^2} = 1-x^2+x^4 - \dots$ при $|x|<1$. Непонятно, почему в $1$ проблема. Никаких перегибов. Но посмотрим на функцию $\frac1{1+z^2}$. Тогда функция не определена для $z_{1,2} = \pm i$, и как раз $|z_{1,2}| = 1$.
\begin{Sl}[Формула Муавра]
	Если $z = r(\cos\phi + i\, \sin \phi)\ne 0$ и если $n\in\Z$, то $z^n = r^n(\cos n\phi + i\,\sin n\phi)$.
\end{Sl}
\begin{Def}
	Пусть $n\in\N$ и $n\ge 2$. Тогда корнем $n$-й степени из числа $z$ называется множество $\sqrt[n]z = \{w\mid w^n =z\}$, то есть множество решений уравнения $w^n=z$.
\end{Def}
В этом курсе очень часто будут изучаться многозначные функции.

Из формулы Муавра следует, что если $z=0$, то $\sqrt[n]z =0$, а при $z=r(\cos\phi_0 + i\,\sin \phi_0)\ne 0$, где $\phi_0 = \arg(z)$, имеем $\sqrt[n]z = \{w_0,\dots,w_{n-1}\}$, где 
\[
	w_j = \sqrt[n]r\left(\cos\frac{\phi_0+2\pi j}n + i\, \sin\frac{\phi_0+2\pi j}n\right),\quad j = 0,\dots,n-1.
\]
\begin{Task}
	Доказать, что $\RY j0{n-1}w_j=0$ и что других корней нет.
\end{Task}
Рассмотрим пример: $\sqrt[4]{-1} = \left\{
	\frac{1+i}{\sqrt2},
	\frac{-1+i}{\sqrt2},
	\frac{-1-i}{\sqrt2},
	\frac{1-i}{\sqrt2}
\right\}$. Здесь $\phi_0=\pi$.
\begin{The}[Основная теорема алгебры]
	Пусть $p(z) = a_n z^n + \dots + a_0$ "--- многочлен (от $z$) степени $n\ge 1$, то есть $a_n\ne 0$, $a_j\in\C$. Тогда $\exists\ z_0\in C\colon p(z_0) = 0$. Говорят, что поле $\C$ алгебраически замкнуто.
\end{The}
Эта теорема у вас уже была доказана, но мы потом независимо докажем.

Поле $\R$ алгебраически замкнутым не является.

Из теоремы следует, что $p(z) = a_n(z-z_1)(z-z_2)\dots(z-z_{n})$.

Оказывается, что другого такого обобщения для $\R$ нет.
\begin{The}[Фробениуса]
	Пусть $P$ "--- поле, содержащее $\R$. И пусть размерность $p = \dim\limits_{\R} P<+\infty$ (размерность относительно $\R$ означает размерность поля, как линейного пространства над полем $\R$). Тогда утверждается, что либо $p=1$ и $P=\R$, либо $p=2$ и $P\cong \C$.
\end{The}
Доказательство есть на полутора страничках. Мы его опустим.
\subsection{Топология и метрика}
У нас всё-таки анализ. Поэтому пора вводить функции и топологию.
\begin{Def}
	Пусть $z_{1,2}\in\C$. Расстоянием между $z_1$ и $z_2$ называется
	\[
	d(z_1,z_2) = |z_1-z_2| = \sqrt{(x_1-x_2)^2+(y_1-y_2)^2},
	\]
	как и в $\R^2$.
\end{Def}
Все свойства расстояния работают, как и в $\R^2$.
\begin{Def}
	Пусть $z_0\in\C$ и $\delta>0$. Открытый круг с центром $z_0$ и радиусом $\delta$ будем обозначать
	\[
		B(z_0,\delta) = \big\{z\in\C\big||z-z_0|<\delta\big\}
	\] и называть $\delta$-окрестностью точки $z_0$.
\end{Def}
Как только есть понятие окрестности, возникают понятия открытого, замкнутого, ограниченного, компактного и связного множеств. Напомню кое-что.
\begin{Def}
	Множество $X\subset\C\colon X\ne \q$ является несвязным, если $\exists$ открытые $U_1,U_2\subset\C$, для которых
	\begin{itemize}
		\item $U_1\cap X\ne\q$;
		\item $U_2\cap X\ne \q$;
		\item $U_1\cap U_2 = \q$;
		\item $X\subset U_1\cup U_2$.
	\end{itemize}

	Если множество непусто и не является несвязным, то оно называется связным.
\end{Def}

\begin{Def}
	Путём в $\C$ называется всякое непрерывное отображение отрезка $[\alpha,\beta]\subset \R$ в $\C$. При этом $-\infty<\alpha<\beta<+\infty$.
\end{Def}
Так как $[\alpha,\beta]\subset\R$, то на отрезке возникается индуцированная топология. Непрерывность понимается в смысле этой топологии. Обозначения следующие
\[
	\gamma\colon [\alpha,\beta]\to\C,\quad\gamma(t).
\]
\begin{Def}
	Множество $X\subset \C$ называется линейно связным, если $\forall\ z_1,z_2\in X\pau \exists$ путь $\gamma\colon[\alpha,\beta]\to\C$, для которого
	\begin{roItems}
	\item $[\gamma]:=\gamma\big([\alpha,\beta]\big)\subset X$\footnote{$[\gamma]$ называется носителем пути или траекторией.};
	\item $\gamma(\alpha)=z_1$, $\gamma(\beta)=z_2$.
	\end{roItems}
\end{Def}
\begin{Task}
	Если $U\ne\q$, то $U$ связно, если и только если $U$ линейно связно.
\end{Task}
\begin{Def}
	Всякое открытое и связное множество называется областью.
\end{Def}
\begin{figure}[H]
	\centering
	\begin{tikzpicture}[xscale=5]
		\datavisualization[school book axes, visualize as smooth line]
		data[format=function] {
			var x : interval [0.01:1] samples 333;
			func y = sin(1/(\value x) r);
		};
		\draw (0.5,1) node[anchor=south]{$\sin\frac1x$};
	\end{tikzpicture}
	\caption{Связное, но не линейносвязное множество}
	\label{fig2}
\end{figure}
\begin{Def}
	Пусть есть последовательность $\pos z\subset\C$. Говорят, что последовательность сходится $c\in C$, если
	\[
		\yop z_n = c \iff \forall\ \e>0\pau\exists\ N\in\N\colon \forall\ n>N\pau |z_n-z|<\e.
	\]
\end{Def}
\begin{Task}
	Пусть $z_n = x_n + i\, y_n$, а $c = a+i\, b$. Тогда $\yop z = c\iff
	\begin{cases}
		x_n\te a;\\y_n\te b.
	\end{cases}$
\end{Task}
\begin{Task}
	То же самое, только для тригонометрической формы.
	\begin{roItems}
	\item Пусть $c=0$. Тогда $\yop z_1=c\iff \yop|z_n|=0$.
	\item Пусть $c\ne 0$ и $c = \underbrace{|c|}_{\rho>0}(\cos\psi+i\,\sin\psi)$, $\psi\in \Arg(c)$. Тогда $z_n\te c$, где $z_n = r_n(\cos\phi_n+i\,\sin \phi_n)$, если и только если
		\[
			\begin{cases}
				r_n\te \rho;\\\phi_n\te \psi\pmod{2\pi}
			\end{cases}
		\]
	\end{roItems}
	Запись $\phi_n\te \psi\pmod{2\pi}$ важна только для $\psi = \pi$.
\end{Task}
